\documentclass[12pt]{beamer}
\usetheme{Boadilla}
\usepackage[utf8]{inputenc}
\usepackage{amsmath, amsfonts, tikz, amssymb}
\usepackage{comment}
\usepackage{blkarray}
\author{Kalyani Gadgil}
\title{Compressed Sparse Column (CSC) Storage }
%\setbeamercovered{transparent} 
%\setbeamertemplate{navigation symbols}{} 
%\logo{} 
%\institute{} 
%\date{} 
%\subject{} 
\begin{document}
\newcommand\hl[1]{\tikz[overlay, remember picture,baseline=-\the\dimexpr\fontdimen22\textfont2\relax]\node[rectangle,fill=green!60,rounded corners,fill opacity = 0.2,draw,thick,text opacity =1] {$#1$};} 

\newcommand{\tikzmark}[2]{\tikz[overlay, remember picture] \node[inner sep=0pt, outer sep=0pt, anchor=base] (#1) {#2};}

\setbeamerfont{footnote}{size=\tiny}
 
\begin{frame}
\titlepage
\end{frame}

%\begin{frame}
%\tableofcontents
%\end{frame}
\begin{frame}{Need for Efficient Sparse Matrix Storage}
	\begin{center}
		Sparse matrices contain a lot of zeros and not all of them are useful information. The most common way is to use the CSR/CSC format.
		\end{center}
\end{frame}

\begin{frame}{Storage Complexity of Sparse Matrices for SUMMA}
	\begin{enumerate}
	\item Blocks of size $(n/\sqrt{p})\times (n/\sqrt{p})$ \\ where n - matrix dim and p - number of processors
	\item Storing each of these submatrices in CSC format $O(n\sqrt{p} ++ nnz)$
	\item Storing whole matrix $O(n + nnz)$ on single processor
	\item Storing matrix in DCSC format requires $O(nnz)$
\end{enumerate}		

\end{frame}


\begin{frame}[fragile]{CSC example - Column Major}
\begin{columns}
\begin{column}{0.5\textwidth}
  \centerline{Matrix A} \\
   \begin{blockarray}{ccccccc}
	\hspace{1cm} & 0 & 1 & 2 & 3 & 4 & 5 \\
\begin{block}{c(cccccc)}
  0 & 0 & 1 & 0 & 0 & 0 & 0\\
  1 & 0 & 0 & 0 & 0 & 0 & 0\\
  2 & 0 & 0 & 3 & 0 & 0 & 0\\
  3 & 0 & 2 & 0 & 0 & 0 & 0\\
  4 & 0 & 0 & 0 & 0 & 0 & 0\\
  5 & 0 & 0 & 0 & 0 & 0 & 4\\
\end{block}
\end{blockarray}

\begin{comment}
	\begin{tabular}{c|c|c|c|c|c|c|c|c|c|}
	\hline
	0 & 1 & 0 & 0 & 0 & 0 & 0 \\
	\hline 
	1 & 0 & 0 & 2 & 0 & 0 & 0 \\
	\hline 
	2 & 0 & 0 & 0 & 3 & 0 & 0 \\
	\hline
	3 & 0 & 0 & 0 & 0 & 0 & 0 \\
	\hline
	4 & 0 & 0 & 0 & 4 & 5 & 0 \\
	\hline
	5 & 0 & 0 & 0 & 0 & 0 & 0 \\
	\hline
	\end{tabular}\\
\end{comment}

\end{column}
\begin{column}{0.5\textwidth}  %%<--- here
\begin{center}
	Column-major \\
	nnz = 4 \\
	NUM = [1, 2, 3, 4] \\
    row idx (IR) = [0, 3, 2, 5] \\ 
    col idx = [1, 1, 2, 5] \\
\end{center}
	
\end{column}
\end{columns}
\end{frame}

\begin{frame}[fragile]{CSC - Walkthrough}
\begin{columns}
\begin{column}{0.5\textwidth}
  \centerline{Matrix A} \\
   \begin{blockarray}{ccccccc}
	\hspace{1cm} & \hl{0} & 1 & 2 & 3 & 4 & 5 \\
\begin{block}{c(cccccc)}
  0 & 0 & 1 & 0 & 0 & 0 & 0\\
  1 & 0 & 0 & 0 & 0 & 0 & 0\\
  2 & 0 & 0 & 3 & 0 & 0 & 0\\
  3 & 0 & 2 & 0 & 0 & 0 & 0\\
  4 & 0 & 0 & 0 & 0 & 0 & 0\\
  5 & 0 & 0 & 0 & 0 & 0 & 4\\
\end{block}
\end{blockarray}

\end{column}
\begin{column}{0.5\textwidth}  %%<--- here
\begin{center}
	JC stores indices of NUM where column changes	\\
	idx of NUM = [0] \\
	\vspace{1cm}
	NUM = [\hspace{1cm} \hl{$-$} \hspace{1cm}] \\
	\vspace{1cm}
    IR  = [\hspace{1cm}\hl{$-$}\hspace{1cm}] \\ 
	\vspace{1cm}
	Column changed, so idx stored \\    
    JC = [\hspace{1cm}\hl{0}\hspace{1cm}] \\
\end{center}
	
\end{column}
\end{columns}
\end{frame}

\begin{frame}[fragile]{CSC example}
\begin{columns}
\begin{column}{0.5\textwidth}
  \centerline{Matrix A} \\
   \begin{blockarray}{ccccccc}
	\hspace{1cm} & 0 & \hl{1} & 2 & 3 & 4 & 5 \\
\begin{block}{c(cccccc)}
  0 & 0 & \hl{1} & 0 & 0 & 0 & 0\\
  1 & 0 & 0 & 0 & 0 & 0 & 0\\
  2 & 0 & 0 & 3 & 0 & 0 & 0\\
  3 & 0 & 2 & 0 & 0 & 0 & 0\\
  4 & 0 & 0 & 0 & 0 & 0 & 0\\
  5 & 0 & 0 & 0 & 0 & 0 & 4\\
\end{block}
\end{blockarray}

\end{column}
\begin{column}{0.5\textwidth}  %%<--- here
\begin{center}
	JC stores indices of NUM where column changes	\\	
	idx of val = [0] \\
	\vspace{1cm}
	NUM = [\hspace{1cm}\hl{1}\hspace{1cm}] \\
	\vspace{1cm}
    IR  = [\hspace{1cm}\hl{0}\hspace{1cm}] \\ 
	\vspace{1cm}
	Column changed, so idx stored \\
    JC = [0,\hspace{1cm}\hl{0}\hspace{1cm}] \\
\end{center}
	
\end{column}
\end{columns}
\end{frame}

\begin{frame}[fragile]{CSC example}
\begin{columns}
\begin{column}{0.5\textwidth}
  \centerline{Matrix A} \\
   \begin{blockarray}{ccccccc}
	\hspace{1cm} & 0 & \hl{1} & 2 & 3 & 4 & 5 \\
\begin{block}{c(cccccc)}
  0 & 0 & 1 & 0 & 0 & 0 & 0\\
  1 & 0 & 0 & 0 & 0 & 0 & 0\\
  2 & 0 & 0 & 3 & 0 & 0 & 0\\
  3 & 0 & \hl{2} & 0 & 0 & 0 & 0\\
  4 & 0 & 0 & 0 & 0 & 0 & 0\\
  5 & 0 & 0 & 0 & 0 & 0 & 4\\
\end{block}
\end{blockarray}

\end{column}
\begin{column}{0.5\textwidth}  %%<--- here
\begin{center}
	JC stores indices of NUM where column changes	\\	
	idx of val = [0, 1] \\
	\vspace{1cm}
	NUM = [1, \hspace{1cm}\hl{2}\hspace{1cm}] \\
	\vspace{1cm}
    IR  = [0, \hspace{1cm}\hl{3}\hspace{1cm}] \\ 
	\vspace{1cm}
	Column not changed, so idx not stored \\
    JC = [0, 0, \hspace{1cm}\hl{}\hspace{1cm}] \\
\end{center}
	
\end{column}
\end{columns}
\end{frame}

\begin{frame}[fragile]{CSC example}
\begin{columns}
\begin{column}{0.5\textwidth}
  \centerline{Matrix A} \\
   \begin{blockarray}{ccccccc}
	\hspace{1cm} & 0 & 1 & \hl{2} & 3 & 4 & 5 \\
\begin{block}{c(cccccc)}
  0 & 0 & 1 & 0 & 0 & 0 & 0\\
  1 & 0 & 0 & 0 & 0 & 0 & 0\\
  2 & 0 & 0 & \hl{3} & 0 & 0 & 0\\
  3 & 0 & 2 & 0 & 0 & 0 & 0\\
  4 & 0 & 0 & 0 & 0 & 0 & 0\\
  5 & 0 & 0 & 0 & 0 & 0 & 4\\
\end{block}
\end{blockarray}

\end{column}
\begin{column}{0.5\textwidth}  %%<--- here
\begin{center}
	JC stores indices of NUM where column changes	\\	
	idx of val = [0, 1, 2] \\
	\vspace{1cm}
	NUM = [1, 2, \hspace{1cm}\hl{3}\hspace{1cm}] \\
	\vspace{1cm}
    IR  = [0, 3, \hspace{1cm}\hl{2}\hspace{1cm}] \\ 
	\vspace{1cm}
	Column changed, so idx stored \\
    JC = [0, 0, \hspace{1cm}\hl{2}\hspace{1cm}] \\
\end{center}
	
\end{column}
\end{columns}
\end{frame}

\begin{frame}[fragile]{CSC example}
\begin{columns}
\begin{column}{0.5\textwidth}
  \centerline{Matrix A} \\
   \begin{blockarray}{ccccccc}
	\hspace{1cm} & 0 & 1 & 2 & \hl{3} & 4 & 5 \\
\begin{block}{c(cccccc)}
  0 & 0 & 1 & 0 & 0 & 0 & 0\\
  1 & 0 & 0 & 0 & 0 & 0 & 0\\
  2 & 0 & 0 & 3 & 0 & 0 & 0\\
  3 & 0 & 2 & 0 & 0 & 0 & 0\\
  4 & 0 & 0 & 0 & 0 & 0 & 0\\
  5 & 0 & 0 & 0 & 0 & 0 & 4\\
\end{block}
\end{blockarray}

\end{column}
\begin{column}{0.5\textwidth}  %%<--- here
\begin{center}
	JC stores indices of NUM where column changes	\\	
	idx of val = [0, 1, 2, 3] \\
	\vspace{1cm}
	NUM = [1, 2, 3, \hspace{1cm}\hl{-}\hspace{1cm}] \\
	\vspace{1cm}
    IR  = [0, 3, 2, \hspace{1cm}\hl{-}\hspace{1cm}] \\ 
	\vspace{1cm}
	Column changed, so idx stored \\
    JC = [0, 0, 2, \hspace{1cm}\hl{3}\hspace{1cm}] \\
\end{center}
	
\end{column}
\end{columns}
\end{frame}

\begin{frame}[fragile]{CSC example}
\begin{columns}
\begin{column}{0.5\textwidth}
  \centerline{Matrix A} \\
   \begin{blockarray}{ccccccc}
	\hspace{1cm} & 0 & 1 & 2 & 3 & \hl{4} & 5 \\
\begin{block}{c(cccccc)}
  0 & 0 & 1 & 0 & 0 & 0 & 0\\
  1 & 0 & 0 & 0 & 0 & 0 & 0\\
  2 & 0 & 0 & 3 & 0 & 0 & 0\\
  3 & 0 & 2 & 0 & 0 & 0 & 0\\
  4 & 0 & 0 & 0 & 0 & 0 & 0\\
  5 & 0 & 0 & 0 & 0 & 0 & 4\\
\end{block}
\end{blockarray}

\end{column}
\begin{column}{0.5\textwidth}  %%<--- here
\begin{center}
	JC stores indices of NUM where column changes	\\	
	idx of val = [0, 1, 2, 3] \\
	\vspace{1cm}
	NUM = [1, 2, 3, \hspace{1cm}\hl{-}\hspace{1cm}] \\
	\vspace{1cm}
    IR  = [0, 3, 2, \hspace{1cm}\hl{-}\hspace{1cm}] \\ 
	\vspace{1cm}
	Column changed, so idx stored \\
    JC = [0, 0, 2, 3, \hspace{1cm}\hl{3}\hspace{1cm}] \\
\end{center}
	
\end{column}
\end{columns}
\end{frame}

\begin{frame}[fragile]{CSC example}
\begin{columns}
\begin{column}{0.5\textwidth}
  \centerline{Matrix A} \\
   \begin{blockarray}{ccccccc}
	\hspace{1cm} & 0 & 1 & 2 & 3 & 4 & \hl{5} \\
\begin{block}{c(cccccc)}
  0 & 0 & 1 & 0 & 0 & 0 & 0\\
  1 & 0 & 0 & 0 & 0 & 0 & 0\\
  2 & 0 & 0 & 3 & 0 & 0 & 0\\
  3 & 0 & 2 & 0 & 0 & 0 & 0\\
  4 & 0 & 0 & 0 & 0 & 0 & 0\\
  5 & 0 & 0 & 0 & 0 & 0 & \hl{4}\\
\end{block}
\end{blockarray}

\end{column}
\begin{column}{0.5\textwidth}  %%<--- here
\begin{center}
	JC stores indices of NUM where column changes	\\	
	idx of val = [0, 1, 2, 3] \\
	\vspace{1cm}
	NUM = [1, 2, 3, \hspace{1cm}\hl{4}\hspace{1cm}] \\
	\vspace{1cm}
    IR  = [0, 3, 2, \hspace{1cm}\hl{5}\hspace{1cm}] \\ 
	\vspace{1cm}
	Column changed, so idx stored \\
    JC = [0, 0, 2, 3, 3, \hspace{1cm}\hl{3}\hspace{1cm}] \\
\end{center}
	
\end{column}
\end{columns}
\end{frame}

\begin{frame}[fragile]{CSC example}
\begin{columns}
\begin{column}{0.5\textwidth}
  \centerline{Matrix A} \\
   \begin{blockarray}{ccccccc}
	\hspace{1cm} & 0 & 1 & 2 & 3 & 4 & 5 \\
\begin{block}{c(cccccc)}
  0 & 0 & \hl{1} & 0 & 0 & 0 & 0\\
  1 & 0 & 0 & 0 & 0 & 0 & 0\\
  2 & 0 & 0 & \hl{3} & 0 & 0 & 0\\
  3 & 0 & \hl{2} & 0 & 0 & 0 & 0\\
  4 & 0 & 0 & 0 & 0 & 0 & 0\\
  5 & 0 & 0 & 0 & 0 & 0 & \hl{4}\\
\end{block}
\end{blockarray}

\end{column}
\begin{column}{0.5\textwidth}  %%<--- here
\begin{center}
	JC stores indices of NUM where column changes	\\	
	idx of val = [0, 1, 2, 3] \\
	\vspace{1cm}
	NUM = [1, 2, 3, 4] \\
	\vspace{1cm}
    IR  = [0, 3, 2, 5] \\ 
	\vspace{1cm}
	NNZ stored as the n+1 element\\
	\vspace{1cm}
    JC = [0, 0, 2, 3, 3, 3,\hspace{0.5cm} \hl{4}\hspace{0.5cm}] \\
\end{center}
	
\end{column}
\end{columns}
\end{frame}






\begin{frame}[fragile]{CSC example - Column Major}
	\begin{enumerate}
		\item $O(n + nnz)$ comes from JC array of size $n+1$
	\end{enumerate}
	\footnotetext{[1] Buluc, A., \& Gilbert, J. R. (2008). On the Representation and Multiplication of Hypersparse Matrices. https://doi.org/10.1109/IPDPS.2008.4536313}
\end{frame}

\begin{frame}{References}
	[1] Buluc, A., \& Gilbert, J. R. (2008). On the Representation and Multiplication of Hypersparse Matrices. https://doi.org/10.1109/IPDPS 2008.4536313 \\
	%[2]http://netlib.org/linalg/html_templates/node91.html
\end{frame}

\end{document}
